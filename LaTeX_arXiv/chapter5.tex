\chapter{Conclusions and and Future Directions} 
\vspace{10mm}

\begin{quote}
  {\it ``I conclude that musical notes and rhythms were first acquired
    by the male and female progenitors of mankind for the sake of
    charming the opposite sex.''} --- Charles Darwin
\end{quote}


\vspace{7mm}
\section{Automatic Music Transcription}
\vspace{5mm}

The most immediate application of the ideas behind {\it Riddim}
is in an automatic music transcription system.  Traditionally, 
transcription involves writing down the notes occurring in a
piece of music, converting an acoustic signal to a symbolic
representation \cite{Klapuri:98}. Learning to 
transcribe music involves however, a lifetime of musical training
and specific training in the styles to be transcribed. An
automatic transcription software tool thus saves time 
and is useful for musicians and composers for a variety of 
analytic and pedagogical purposes. For example, in many modern styles 
where a score is not available, a symbolic representation of what is 
occurring in the music is very valuable. Similarly, Jazz buffs trying 
to learn the intricacies of an improvised jazz solo can benefit from 
such a system as well as can academic composers trying to analyse an 
electro-acoustic work.

To make {\it Riddim} a robust music transcription system will 
entail additional work. First, a fundamental frequency 
tracking module is essential for a classic ``score'' transcriber. 
This module would essentially traverse the extracted streams 
performing a pitch estimation at every onset point. In this
way, the onset times would correspond to the rhythmic part
of the score while the estimated fundamental frequency would
approximate the pitch. Invariably, as audio analysis techniques become more sophisticated,
automatic music transcription systems like speech recognition
systems are going to be increasingly important in the way people
learn and interact with music. As Klapuri points out, ``Some people
would certainly appreciate a radio receiver capable of tracking
jazz music simultaneously on all channels'' \cite{Klapuri:98}.

\vspace{7mm}
\section{{\it Riddim} in Real-time}
\vspace{5mm}

One of the central motivations of this thesis was to take the music
that I have composed from the studio to the stage, without compromising 
the complexity. At the same time, moving to a more improvisational 
performance setting requires a certain mastery of the musical content 
and the tools to be able to compose on the fly. The current work is of
course non-realtime. Working with the algorithms and the code for the
past nine months, I have determined that a realtime version is not
only feasible, but with minimal additional work, can run as a
MAX/MSP{\texttrademark} external DSP module or as a VST{\texttrademark} plugin.

\vspace{7mm}
\section{Live Improvised Performance and the Meta DJ}
\vspace{5mm}

Once running in real-time, what are its musical applications? 
In a live performance setting where listening and playing within 
a shared musical context is paramount, {\it Riddim} ``listens'' to the surrounding
music, analyses it and reveals patterns that can be used time a new
instrument or part. How these patterns are mixed with the original
material is a delicate question of production and aesthetics, as the
choice of voicing makes a world of difference.

Furthermore, in analysing and re-interpreting music on the fly, mixing
elements occurs in an abstract representation of the music rather than the 
common DJ practise of mixing the concrete representation of the music i.e. samples.
The idea of a {\it Meta DJ} involves mixing an abstract representation
of ``vinyl cuts'' or CD tracks by synchronising the realtime outputs of {\it Riddim} and
rendering the patterns with variety of instruments appropriate for the
musical setting. The flexibility of not having to mix concrete samples
is significant since resampling, time-stretching and vocoder artifacts are common
byproducts trying to match two tracks at different tempi. Furthermore, the time patterns
extracted in realtime can be used to control light shows, smoke
machines or any number of patterned sensory stimuli that accompany the
music. 



